\documentclass[14pt]{article}
%Fonts and Math packages
\usepackage{amsmath,amssymb,amsfonts,amsthm,bm,bbm,cancel,wasysym}
%Graphics packages
\usepackage{epsfig,graphics,graphicx,epstopdf}
%Tabular packages
\usepackage{array,booktabs,colortbl,colordvi,multirow}
% Colors
\usepackage{colordvi,color,xcolor}
% Links
\usepackage{hyperref}
% Others
\usepackage{rotating}
\usepackage{authblk}

%This is the next version file

%------------------------------ TOM'S preamble --------------------
\usepackage{verbatim}
\usepackage{cite}
\usepackage{subfig}
\usepackage{setspace}
\usepackage{url}
\usepackage[percent]{overpic}
\usepackage{slashed}
\usepackage{authblk}
\usepackage{xspace}
\usepackage{fullpage}
\usepackage{hyperref}

% Commonly used things
\newcommand{\Fermi}[0]{\textit{Fermi}\xspace}
\newcommand{\gtlike}[0]{\lstinline!gtlike!\xspace}
\newcommand{\sigmav}[0]{\ensuremath{\langle \sigma v \rangle}\xspace}
\newcommand{\bbbar}[0]{\ensuremath{b \bar b}\xspace}

\newcommand{\sigv}[0]{\ensuremath{\langle\sigma v\rangle}\xspace}
\newcommand{\sigvm}[0]{\ensuremath{\langle\sigma v\rangle_{\rm{UL}}}\xspace}
\newcommand{\tsigv}[0]{\ensuremath{\langle\sigma v\rangle R^{2}}\xspace}

% Units
\newcommand{\unit}[1]{\ensuremath{\mathrm{\,#1}}\xspace}
\newcommand{\TeV}{\unit{TeV}}
\newcommand{\GeV}{\unit{GeV}}
\newcommand{\MeV}{\unit{MeV}}
\newcommand{\degree}{\unit{^{\circ}}}
\newcommand{\cm}{\unit{cm}}
\newcommand{\kpc}{\unit{kpc}}
\newcommand{\second}{\unit{s}}
\newcommand{\photons}{\unit{photons}}
\renewcommand{\photon}{\unit{photons}}
\newcommand{\ph}{\unit{photons}}
\newcommand*{\pt}{\ensuremath{p_{\text{T}}}}

% Reference
\newcommand{\Figref}[1]{Figure \ref{figs:#1}}
\newcommand{\Secref}[1]{Section \ref{sec:#1}}
\newcommand{\Subsecref}[1]{Section \ref{subsec:#1}}
\newcommand{\Eqnref}[1]{Eq.~(\ref{eqn:#1})}

\newcommand{\figref}[1]{figure \ref{figs:#1}}
\newcommand{\secref}[1]{section \ref{sec:#1}}
\newcommand{\subsecref}[1]{section \ref{subsec:#1}}
\newcommand{\eqnref}[1]{eq.~(\ref{eqn:#1})}

\def\tev{\,{\rm TeV}}
\def\gev{\,{\rm GeV}}
\def\ie{{\it i.e.}}
\def\eg{{\it e.g.}}
\def\etc{{\it etc}}
\def\etal{{\it et al.}}
\def\ibid{{\it ibid}.}

\def\to{\rightarrow}
\def\Fermi{\,{\it Fermi}}

\newskip\zatskip \zatskip=0pt plus0pt minus0pt
\def\matth{\mathsurround=0pt}
\def\lsim{\mathrel{\mathpalette\atversim<}}
\def\gsim{\mathrel{\mathpalette\atversim>}}
\def\atversim#1#2{\lower0.7ex\vbox{\baselineskip\zatskip\lineskip\zatskip
  \lineskiplimit 0pt\ialign{$\matth#1\hfil##\hfil$\crcr#2\crcr\sim\crcr}}}


%------------------------------Comments --------------------

\newcommand{\JH}[1]{{\color{green} \bf JLH~\![#1]}} % J. Hewett comments

\newcommand{\TR}[1]{{\color{red} \bf TR~\![#1]}} % T. Rizzo comments


%--------------------------- TEXT DIMENSIONS AND MARGINS ------------------------------%

\renewcommand{\topfraction}{1}
\renewcommand{\bottomfraction}{1}
\renewcommand{\textfraction}{0}
\renewcommand{\baselinestretch}{1.0}
\parskip=1.0ex
\setcounter{topnumber}{10}
\setcounter{bottomnumber}{10}
\setcounter{totalnumber}{10}

\textwidth=16cm
\textheight=23cm
\oddsidemargin=0.2cm
\evensidemargin=0.2cm

\author[1]{Clement Helsens}
\author[2]{David Jamin}
\author[3]{Michelangelo L. Mangano}
\author[4]{Thomas G. Rizzo}
\author[1]{Michele Selvaggi}

\affil[1]{CERN EP-Departement, CH-1211 Geneva 23, Switzerland}
\affil[2]{Academia Sinica, Institute of  Physics, Taipei, Taiwan}
\affil[3]{CERN TH-Departement, CH-1211 Geneva 23, Switzerland}
\affil[4]{SLAC National Accelerator Laboratory 2575 Sand Hill Rd., Menlo Park, CA, 94025 USA}

\title{$Z'$ identification at the HE-LHC} 

%------------------------------------- BEGIN DOCUMENT ------------------------------------------%

\begin{document}
\maketitle
\tableofcontents


%--------------------------------------------- ABSTRACT ---------------------------------------------%
\begin{abstract}
 
\noindent

The 14 TeV LHC with L=3 ab$^{-1}$ of integrated luminosity can search for new $Z'$ gauge bosons from the classic extended gauge theories up to masses 
of roughly $\simeq 6$ TeV.  Here we analyzed the capability of the 27 TeV HE-LHC with L=15 ab$^{-1}$ to distinguish among six $Z'$ models employing only 
the $e^+e^-$ and $\mu^+\mu^-$ channels assuming that $M_{Z'}=6$ TeV. Under the assumption that these $Z'$'s decay only to SM particles,  we show that 
there are sufficient observables to perform this model differentiation in most cases. 
\end{abstract}


%-------------------------------- DOCUMENT: INTRODUCTION ---------------------------------%

\section{Context of the study}
It is legitimate to assume that a heavy resonance could be seen at the end of High Luminosity LHC (HL-HLC). If that is the case a new collider with higher energy 
in the center of mass is needed to study its property as not enough events will be available at 14~TeV. In this document we present the discrimination potential of a High Energy LHC (HE-LHC)
with an assumed center of mass energy of 27~TeV.


\section{Bounds from HL-LHC}
As a starting point it is needed to understand what are, for $\sqrt s=14$ TeV, for the typical exclusion/discovery reaches for some standard reference $Z'$ models assuming L=3 ab$^{-1}$ 
employing only the $e^+e^-$ and $\mu^+\mu^-$ channels. To address this and the other questions below we will use the same set of $Z'$ models as employed 
in \cite{Rizzo:2014xma} and mostly in \cite{Han:2013mra}, both of which we will refer to frequently. We employ the MMHT2014 NNLO set \cite{Harland-Lang:2014zoa} 
throughout with an appropriate constant $K$-factor (=1.27) for numerics. 
The upper panel in Fig.~\ref{toy} shows the production cross section times leptonic BF for these models at 14 TeV in the NWA. It has 
been and will be assumed here that these $Z'$ states only decay to SM particles. 

Using the present ATLAS and CMS results at 13 TeV, \cite{Aaboud:2017buh} and \cite{Sirunyan:2018exx}, it is straightforward to estimate by extrapolation the eventual 14 TeV 
exclusion reach in the combined $e+\mu$ sample; this is given in the first column of Table~\ref{spec}. For discovery, only the $e$ channel is used due to poor $\mu$-pair mass 
resolution near $M_{Z'}=6$ TeV although the muons will add some additional support for any observed excess as large masses. Estimates of the $3\sigma$ evidence and $5\sigma$ 
discovery limits are also given in the Table. Based on these results, we will assume in our study below that we are dealing with a $Z'$ of mass 6 TeV; somewhat smaller mass 
choices will lead to very similar conclusions based on the ratio of predicted event rates shown in the lower panel of Fig.~\ref{toy}.  Fig.\ref{toy2} shows the NWA cross 
sections for the same set of models but now at 27 TeV; with L=15 ab$^{-1}$, we note that very large statistical samples will be available for the case of $M_{Z'}=6$ TeV 
for each dilepton channel. 


\begin{figure}[htbp]
  \centering
    \includegraphics[trim={2cm 2cm 2cm 2cm},clip,width=0.49\columnwidth]{figures/zp14tev-ref.pdf}
    \includegraphics[trim={2cm 2cm 2cm 2cm},clip,width=0.49\columnwidth]{figures/scaled-ratio.pdf}
\caption{(Top) $\sigma B_l$ in the NWA for the $Z'$ production at the $\sqrt s=14$ TeV LHC as functions of the $Z'$ mass: SSM(red), LRM (blue), $\psi$(green), $\chi$(magenta), 
$\eta$(cyan), I(yellow).  (Bottom) Ratio of the number of events  for $\sqrt s=27$ TeV, L=15 ab$^{-1}$ to that at 13[14] TeV, L=3 ab$^{-1}$ for 
$\bar u u$ (red) and $\bar d d$ (blue)  [green, cyan] initial state partons with fixed invariant mass $M$.}
\label{toy}
\end{figure}



%
\begin{table}
\centering
\begin{tabular}{|l|c|c|c|} \hline\hline
  Model &   95$\%$ CL     &  $3\sigma$     &   $5\sigma$   \\
\hline
SSM    &     6.62     &  6.09        &  5.62     \\
LRM    &   6.39     & 5.85        & 5.39  \\
$\psi$    &  6.10   & 5.55   & 5.07  \\
$\chi$   &  6.22    & 5.68    & 5.26   \\
$\eta$   &  6.15     &  5.59  &  5.16   \\
~I        & 5.98   &  5.45   &  5.05  \\
\hline\hline
\end{tabular}
\caption{ $\sqrt s=14$ TeV results for $M_{Z'}$ in TeV as discussed in the text. }
\label{spec}
\end{table}
%


\section{Discrimination from direct calculations}
Question: Can we use just this dilepton channel to distinguish between these six $Z'$ models? 
\newline
\\
We make use of 3 observables, all in NWA: $\sigma B_l$, the forward-backward asymmetry, $A_{FB}$ and the rapidity ratio, $r_y$. These last two are defined 
and discussed at some extent in both \cite{Rizzo:2014xma} and \cite{Han:2013mra}.  Given the ATLAS and CMS analyses as presented in \cite{Han:2016qpd} 
and \cite{CMS:2017zzj} we employ the entire range of rapidity $|y|\leq 2.5$ in defining these quantities. Based on these same works and \cite{Han:2013mra}, 
In addition to usual statistical errors, we will assume a $2\%$ systematic error on the two ratios as most uncertainties (lumi and PDF) will cancel between numerators 
and denominators. For $\sigma B_l$, we assign a $5\%$ systematic error in addition to this $2\%$; all errors are then added in quadrature. I am aware that these may 
be aggressive numbers but the plots will tell all. 


\begin{figure}[htbp]
  \centering
\includegraphics[trim={2cm 2cm 2cm 2cm},clip,width=0.49\columnwidth]{figures/zp27tev-ref.pdf}
\caption{ Same as the top panel in the previous Figure but now for the HE-LHC.}
\label{toy2}
\end{figure}



Fig.~\ref{toy3} shows the correlated predictions for these 3 observables for these six models given the above assumptions and employing {\it only} a single dilepton 
channel. Here we see that apart from a possible near degeneracy in models $\psi,\eta$, a reasonable $Z'$ model separation is indeed achieved. It is clear that this 
remains possible even with somewhat larger values for the systematic errors. I note again that the use of $\sigma B_l$ relies on the absence of new non-SM decay 
modes.



\begin{figure}[htbp]
  \centering
\includegraphics[trim={2cm 2cm 2cm 2cm},clip,width=0.49\columnwidth]{figures/compare2.pdf}
\includegraphics[trim={2cm 2cm 2cm 2cm},clip,width=0.49\columnwidth]{figures/compare3.pdf}
\caption{(Top) $\sigma B_l$ vs $A_{FB}$ and (Bottom) $r_y$ vs $A_{FB}$ at the HE-LHC assuming $M_Z'=6$ TeV as discussed in the text. 
SSM(black), LRM(red), $\psi$ (blue), $\chi$ (green), $\eta$ (magenta) and I (cyan). $1\sigma$ errors only are shown. }
\label{toy3}
\end{figure}




\section{Discrimination from detector level analysis}
The analyses presented in this section are all performed within the FCC software framework, FCCSW~\cite{fccsw}.
The detector parametrisation considered in this study is from HE-LHC official parametrisation for the yellow report~\cite{HELHCtwiki}.
The Monte Carlo are first presented in Section~\ref{subsection:MC} Leptonic decays are presented in~\ref{subsection:lepana}, and hadronic analyses in~\ref{subsection:hadana}.


\subsection{Monte Carlo Samples}
\label{subsection:MC}
Monte Carlo~(MC) simulated event samples were used to simulate the response of the FCC detector to signal and backgrounds. The muon momentum resolution 
is assumed to be $\sigma(p)/p \approx 20\%$ at $\pt= 20 $TeV. Signals are generated with {\scshape Pythia}~8.230~\cite{Sjostrand:2014zea} using the leading 
order cross-section from the generator. All lepton flavour decays of the $Z'$ are generated assuming universality of the couplings.
The Drell-Yan background has been generated using {\scshape MG5\_}a{\scshape MC@NLO}~2.5.2~\cite{Alwall:2014hca} at leading order only. 
A k-factor of 2 is applied to all the background processes.



\subsection{Leptonic analysis}
\label{subsection:lepana}
\subsubsection{Event selection and discovery potential}

Events are required to contain at least two same flavour leptons of opposite charge with $\pt > 500$~GeV, $|\eta|<$4.5. 
When mentioned a smaller acceptance of $|\eta|<$2.5 is considered to test the impact on the reduction of the statistics. 
An additional cut on the di-lepton invariant mass $m_{ll}>1$~TeV is applied to remove the low mass Drell-Yan. 
Figure~\ref{figure:lepana:mass} shows the invariant mass for a 6~TeV signal for the $ee$ (left) and $\mu\mu$ channels (right). 
The mass resolution is better for the ee channel, as expected from the higher performance of the electro-magnetic calorimeter at high energy.

\label{sec:lepana}
\begin{figure}[h]
  \centering
    \includegraphics[width=0.45\columnwidth]{figures/Zpmumu_mzp_sel0_nostack_log.eps}
    \includegraphics[width=0.45\columnwidth]{figures/Zpee_mzp_sel0_nostack_log.eps}
   \caption{Invariant mass for a 6~TeV signal after full event selection for ee channel (left) and $\mu\mu$ channel (right).}
  \label{figure:lepana:mass}
\end{figure}

Limits and discovery potential for di-lepton resonances are shown in Figure~\ref{figure:lepana:limdisc}. With the full 15~ab$^{-1}$ 
integrated luminosity, it is expected to exclude heavy $Z'$ resonances from 10 to 12.5~TeV depending on the model and discover 
a $Z'_{SSM}$ up to 12~TeV.

\begin{figure}[!htb]
  \centering
  \includegraphics[width=0.45\columnwidth]{figures/lim_Zprime_ll_helhc_v01_allxs.eps}
  \includegraphics[width=0.45\columnwidth]{figures/DiscoveryPotential_ll_comb_rootStyle.eps}
  \caption{Limit versus mass for the di-lepton channel (left) and luminosity for a $5\sigma$ discovery (right) for the ee and $\mu\mu$ combined channels. }
  \label{figure:lepana:limdisc}
\end{figure}


\subsubsection{Variables definition}
\label{subsubsection:vardef}

As for the analysis from direct calculation, the observables are defined at the detector level. On Figure~\ref{figure:lepana:yzp} the 
rapidity of the $Z'$ is displayed, and as expected it is much more central for the signal than for the Drell-Yan background.

\begin{figure}[!htb]
  \centering
  \includegraphics[width=0.45\columnwidth]{figures/yzp_sel0_lin_norm_ee.eps}
  \includegraphics[width=0.45\columnwidth]{figures/yzp_sel0_lin_norm_mumu.eps}
  \caption{$Z'$ rapidity distribution comparing a 6~TeV signal and Drell-Yan background for ee (left) and $\mu\mu$ channels (right).}
  \label{figure:lepana:yzp}
\end{figure}

At detector level the variable $r_y$ is defined as the ratio of central over forward events:
\begin{equation}
r_y = \frac{\sigma(|y_{Z'}| < y_1)}{\sigma(y_1 < |y_{Z'}| <y_2)}
\end{equation}
with $y_1=0.5$ and $y_2=2.5$ for this study.

The variable $A_{FB}$ can be seen as a measure of the charge asymmetry
\begin{equation}
A_{FB} = A_C =  \frac{\sigma(\Delta|y| > 0) - \sigma(\Delta|y| < 0)}{\sigma(\Delta|y| > 0) + \sigma(\Delta|y| < 0)}
\end{equation}
where $\Delta|y| = |y_l| - |y_{\bar{l}}|$. It has been checked that this definition is equivalent to defining 
\begin{equation}
A_{FB} =   \frac{\sigma_F - \sigma_B}{\sigma_F + \sigma_B}
\end{equation}
with $\sigma_F = \sigma (cos\theta^{*}_{cs})>0$ and $\sigma_B = \sigma (cos\theta^{*}_{cs})<0$ and defining $\theta^*$ in the Collins-Soper 
frame as
\begin{equation}
cos\theta^{*}_{cs} =  \frac{Q_z}{|Q_z|} \frac{2(P_l^+P_{\bar{l}}^- - P_l^-P_{\bar{l}}^+)}{|Q| \sqrt{Q^2+Q^2_T}}
\end{equation}
where $Q$, $Q_T$ and $Q_z$ are the four-momentum, the transverse momentum, and the longitudinal
momentum of the di-lepton pair. $P_{l}(P_{\bar{l}})$ represents the four momenta of the lepton (anti-lepton),
and $P^\pm_l = (P^0_l \pm P^3_l)$.
\newline
impact of eta cut
\newline
impact of ISR/SFR
\newline
show cross section versus int luminosity
\newline
show error on AFB ry for models 

\subsection{Hadronic analysis}
\label{subsection:hadana}
In this section the hadronic decays of the $Z'$ will be tested in order to enhance the discrimination potential. 


\begin{figure}[h]
  \centering
  \includegraphics[width=0.30\columnwidth]{figures/Mj1j2_pf08_MetCorr_fit_sel0_nostack_log_tt.eps}
  \includegraphics[width=0.30\columnwidth]{figures/Mj1j2_pf08_MetCorr_fit_sel0_nostack_log_bb.eps}
  \includegraphics[width=0.30\columnwidth]{figures/Mj1j2_pf08_MetCorr_fit_sel0_nostack_log_jj.eps}
  \caption{Left, center: Invariant mass for a 6~TeV signal after full event selection for ee channel (left) and $\mu\mu$ channel (center). Right: Transverse mass for a 6~TeV signal after full event selection for the $\tau\tau$ channel. }
  \label{figure:leptonicresonances:masses}
\end{figure}

\subsubsection{Results}


\begin{figure}[h]
  \centering
     \includegraphics[width=0.3\columnwidth]{figures/Zp_branching_4TeV_5ab.eps}
    \includegraphics[width=0.3\columnwidth]{figures/Zp_branching_6TeV_5ab.eps}
    \includegraphics[width=0.3\columnwidth]{figures/Zp_branching_8TeV_5ab.eps}

    \includegraphics[width=0.3\columnwidth]{figures/Zp_branching_4TeV_15ab.eps}
    \includegraphics[width=0.3\columnwidth]{figures/Zp_branching_6TeV_15ab.eps}
    \includegraphics[width=0.3\columnwidth]{figures/Zp_branching_8TeV_15ab.eps}

    \includegraphics[width=0.3\columnwidth]{figures/Zp_branching_4TeV_30ab.eps}
    \includegraphics[width=0.3\columnwidth]{figures/Zp_branching_6TeV_30ab.eps}
    \includegraphics[width=0.3\columnwidth]{figures/Zp_branching_8TeV_30ab.eps}
  \caption{Discrimination for 4, 6 and 8TeV (left, center, right) masses with 5, 15 and 30ab$^{-1}$ (up, middle, down) of integrated luminosity. Statistical and full uncertainties are shown on each point.}
  \label{figure:hadana:discri30ab}
\end{figure}

The model discrimination achieved by the hadronic analyses for 4, 6 and 8TeV considering an integrated luminosity of 
5, 15 and 30ab$^{-1}$ respectively is shown on Fig.~\ref{figure:hadana:discri5ab}, \ref{figure:hadana:discri15ab}, \ref{figure:hadana:discri30ab}.
From the leptonic analysis~\ref{sec:lepana}, it was shown that for $r_y$ 
For all cases, except for a mass of 8TeV and an integrated luminosity of 5ab$^{-1}$, discrimination is obtained at more than one sigma for the di-jet case.


\section{Summary}


%------------------------------------ ACKNOWLEDGEMENTS ---------------------------------------%
\section*{Acknowledgements}

%
This work was supported by the Department of Energy, Contract DE-AC02-76SF00515.


%------------------------------------------- REFERENCES -------------------------------------------%


\begin{thebibliography}{99}


%\cite{Rizzo:2014xma}
\bibitem{Rizzo:2014xma} 
  T.~G.~Rizzo,
  %``Exploring new gauge bosons at a 100 TeV collider,''
  Phys.\ Rev.\ D {\bf 89}, no. 9, 095022 (2014)
  doi:10.1103/PhysRevD.89.095022
  [arXiv:1403.5465 [hep-ph]].
  %%CITATION = doi:10.1103/PhysRevD.89.095022;%%
  
%\cite{Han:2013mra}
\bibitem{Han:2013mra} 
  T.~Han, P.~Langacker, Z.~Liu and L.~T.~Wang,
  %``Diagnosis of a New Neutral Gauge Boson at the LHC and ILC for Snowmass 2013,''
  arXiv:1308.2738 [hep-ph].
  %%CITATION = ARXIV:1308.2738;%%
 
 %\cite{Harland-Lang:2014zoa}
\bibitem{Harland-Lang:2014zoa} 
  L.~A.~Harland-Lang, A.~D.~Martin, P.~Motylinski and R.~S.~Thorne,
  %``Parton distributions in the LHC era: MMHT 2014 PDFs,''
  Eur.\ Phys.\ J.\ C {\bf 75}, no. 5, 204 (2015)
  doi:10.1140/epjc/s10052-015-3397-6
  [arXiv:1412.3989 [hep-ph]].
  %%CITATION = doi:10.1140/epjc/s10052-015-3397-6;%%
  

%\cite{Aaboud:2017buh}
\bibitem{Aaboud:2017buh} 
  M.~Aaboud {\it et al.} [ATLAS Collaboration],
  %``Search for new high-mass phenomena in the dilepton final state using 36 fb$^{?1}$ of proton-proton collision data at $ \sqrt{s}=13 $ TeV with the ATLAS detector,''
  JHEP {\bf 1710}, 182 (2017)
  doi:10.1007/JHEP10(2017)182
  [arXiv:1707.02424 [hep-ex]].
  %%CITATION = doi:10.1007/JHEP10(2017)182;%%


%\cite{Sirunyan:2018exx}
\bibitem{Sirunyan:2018exx} 
  A.~M.~Sirunyan {\it et al.} [CMS Collaboration],
  %``Search for high-mass resonances in dilepton final states in proton-proton collisions at $\sqrt{s}=$ 13 TeV,''
  arXiv:1803.06292 [hep-ex].
  %%CITATION = ARXIV:1803.06292;%%


%\cite{Han:2016qpd}
\bibitem{Han:2016qpd} 
  J.~Han [ATLAS and CMS Collaborations],
  %``Dilepton Forward-Backward Asymmetry and electroweak mixing angle at ATLAS and CMS,''
  PoS ICHEP {\bf 2016}, 677 (2016).
 
 %\cite{CMS:2017zzj}
\bibitem{CMS:2017zzj} 
  CMS Collaboration [CMS Collaboration],
  %``Measurement of the weak mixing angle with the forward-backward asymmetry of Drell-Yan events at 8 TeV,''
  CMS-PAS-SMP-16-007.
  %%CITATION = CMS-PAS-SMP-16-007;%%
  
\bibitem{fccsw} FCCSW main page, http://fccsw.web.cern.ch/fccsw/

\bibitem{HELHCtwiki} HE-LHC twiki https://twiki.cern.ch/twiki/bin/view/LHCPhysics/HLHELHCWorkshop

\bibitem{Sjostrand:2014zea}
  T.~Sj�strand {\it et al.},
  %``An Introduction to PYTHIA 8.2,''
  Comput.\ Phys.\ Commun.\  {\bf 191} (2015) 159
  doi:10.1016/j.cpc.2015.01.024
  [arXiv:1410.3012 [hep-ph]].
  %%CITATION = doi:10.1016/j.cpc.2015.01.024;%%
  %1167 citations counted in INSPIRE as of 18 Sep 2018

\bibitem{Alwall:2014hca}
  J.~Alwall {\it et al.},
  %``The automated computation of tree-level and next-to-leading order differential cross sections, and their matching to parton shower simulations,''
  JHEP {\bf 1407} (2014) 079
  doi:10.1007/JHEP07(2014)079
  [arXiv:1405.0301 [hep-ph]].
  %%CITATION = doi:10.1007/JHEP07(2014)079;%%
  %3067 citations counted in INSPIRE as of 18 Sep 2018

\end{thebibliography}

%-------------------------------------------------- END --------------------------------------------------%

\end{document}



  
 


